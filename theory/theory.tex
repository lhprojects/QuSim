    \documentclass[12pt,twoside]{article}
    \usepackage{amsmath}
    \usepackage{amssymb}
    \usepackage{amsthm}
    \usepackage{amsfonts}
    \usepackage{braket}
\usepackage{esvect}

\def\df{{\rm d}}
\begin{document}

\title{LaTeX Typesetting By Example}
\author{Phil Farrell\\
Stanford University School of Earth Sciences}

\renewcommand{\today}{November 2, 1994}
\maketitle

$$
-\hbar^2/(2m) \nabla^2 \psi = i \hbar \partial_t \psi
$$

$$
\psi_p(x) = e^{ i/\hbar (p x - p^2 /(2m)t)}
$$
$$
\Psi(x) = \int dp \phi_x(p) \psi_p(x)
$$

Note
$$
\int dx \psi_{p^\prime}^*(x) \psi_{p}(x) = e^{ i/\hbar ((p^{\prime2} - p^2) /(2m)t)} 2\pi\delta(1/\hbar(p-p^\prime))
=2\pi\hbar\delta(p-p^\prime)
$$

$$
\phi_x(p) = \frac{1}{2\pi\hbar}\int dx \psi^*_p(x) \Psi(x)
$$
the solution for any time:
$$
\Psi(x, t) = \int dp \frac{1}{2\pi\hbar}\int dy \psi^*_p(y) \Psi(y) \psi_p(x,t)
$$

$$
\Psi(x, t) = \int dy  \Psi(y) \int dp \frac{1}{2\pi\hbar}\psi^*_p(y) \psi_p(x,t)
$$

$$
\Psi(x, t) = \int dy  \Psi(y) \int dp \frac{1}{2\pi\hbar} e^{i(p(x-y)/\hbar - p^2 t/(2m\hbar))}
$$

$$
T=t/(m\hbar)
$$
$$
\Psi(x, t) = \int dy  \Psi(y) \frac{1}{2\pi\hbar} (1-i) e^{i (x-y)^2 / (2 T \hbar^2)}  \sqrt{\pi / T}
$$


For space $a$
$$
-\hbar^2/(2m a^2) (\psi(x+a) + \psi(x-a) - 2\psi(x)) \psi = i \hbar \partial_t \psi
$$

$$
\psi_p(x) = e^{ i/\hbar (p x - \hbar^2(1 - \cos(a p/\hbar))/(m a^2)t)}
$$

$$
\psi_p(x) = e^{ i/\hbar (p x - (p^2/(2m))t)}
$$

$$
0 \le p < 2\pi / a
$$

$$
\sum_{x_i} \psi_{p^\prime}^*(x_i) \psi_{p}(x_i) = e^{ i/\hbar (\hbar^2(1 - \cos(a p^\prime/\hbar) - \hbar^2(1 - \cos(a p/\hbar))/(m a^2)t)}
\sum_{x_i}e^{i(p-p^\prime)x_i/\hbar} = 2\pi\delta(1/\hbar(p - p\prime))
$$
$$
x_i = ai
$$

$$
\phi_x(p) = \frac{1}{2\pi\hbar}\sum_{x_i} \psi^*_p(x_i) \Psi(x_i)
$$

$$
\Psi(x, t) = \sum_{y_i} \Psi(y_i) \int dp \frac{1}{2\pi\hbar} e^{ ip(x-y)/\hbar - i\hbar(1 - \cos(a p/\hbar))/(m a^2)t}
$$



$$
g(k) = \int dx f(x) \exp(ikx)
$$

$$
f(x) = \frac{1}{2\pi}\int dk g(k) \exp(-ikx)
$$

$$
f(x) = \sum_j \delta(x - ja)
$$
$$
g(k) = \int dx \sum_j \delta(x - ja) \exp(ikx) = \sum_j \exp(ijka)
$$

$$
f(x) = \sum_j \delta(x - ja) = \frac{1}{2\pi}\int dk g(k) \exp(-ikx)
$$

$$
g(k) = 1/(2\pi a)\sum_n \delta(k+2\pi n/a)
$$


\section{Infinite Wall}

Eigen state
$$
\psi_k(j) = \sin(jk\pi/N)
$$

$$
j=0..N-1
$$

$$
k=1...N-1
$$

Eigen value
$$
E_k = k\pi/(N a)
$$

$$
\sum_j \psi_j(k)\psi_j(k) = N/2
$$

$$
\Psi(j) = \sum_{k=1}^{N-1} \Phi(k) \psi_k(j)
$$

$$
\Phi(k) = (2/N) \sum_j \Psi(j) \psi_k(j)
$$

$$
\Phi(k) = (2/N) \sum_j \Psi(j) \sin(jk\pi/N)
$$

$$
\Phi(k) = (1/(Ni)) \sum_j \Psi(j) (\exp(ijk\pi/N) - \exp(-ijk\pi/N))
$$

$$
\Phi(k) = (1/(Ni)) \sum_j \Psi(j) (\exp(ijk2\pi/2N) - \exp(i(2N-j)k2\pi/2N))
$$

$$
\Phi(k) = \frac{1}{Ni}\left(\sum_j^{N-1} \Psi(j)\exp(ijk2\pi/2N) - \sum_{N+1}^{2N} \Psi(2N-j)\exp(jk2\pi/2N))\right)
$$

by define
$$
\Phi(N)=0
$$

$$
\Phi(k) = \frac{1}{Ni}\left(\sum_j^{N-1} \Psi(j)\exp(ijk2\pi/2N) - \sum_{j=N}^{2N-1} \Psi(2N-j)\exp(ijk2\pi/2N))\right)
$$

$$
\Phi(k) = \frac{1}{Ni}\left(\sum_{j=0}^{2N-1} \Psi^\prime(j)\exp(ijk2\pi/2N)\right)
$$

$$
\Psi^\prime(j) = \Psi(j) (j \le N-1)
$$
$$
\Psi^\prime(j) = 0 (j = N)
$$
$$
\Psi^\prime(j) = -\Psi(2N-j) (j > N)
$$

$$
\Phi(k) = \frac{1}{Ni} {\rm inv fft}(2N, \Psi^\prime(j), k)
$$


$$
\Psi(j) = \sum_{k=1}^{N-1} \Phi(k) \sin(jk\pi/N)
$$

$$
\Psi(j) = \frac{1}{2i} {\rm inv fft}(2N, \Phi^\prime(k), j)
$$

\section{Split Method}

Second order approximation
$$
e^{t+v} = e^{1/2 v} e^{ t} e^{1/2 v}
$$

Fourth-order approximation
$$
e^{t+v} = e^{c_1 v} e^{d_1 t} e^{c_2 v} e^{d_2 t} e^{ c_2 v} e^{ d_1 t} e^{ c_1 v}
$$

where
$$
c_1 = 1/(2(2-2^{1/3}))
$$
$$
c_2 = (1-2^{1/3})/(2(2-2^{1/3}))
$$
$$
d_1 = 1/(2-2^{1/3})
$$
$$
d_2 = -2^{1/3}/(2-2^{1/3})
$$

\section{Eigen Method}

\section{Gauss-Legendre Method}
The SE, can be written as
$$
\frac{d\psi}{ dt} = - I h \psi
$$
the formal solution is
$$
\psi(t) = e^{ -I h t}\psi(0)
$$

In numerical analysis, many methods are special cases of Runge-Kutta methods.
\begin{itemize}
\item Euler Method
\begin{itemize}
    \item Forward Euler Method
    \item Backward Euler Method
\end{itemize}
\item Gauss-Legendre Method
\item ...
\end{itemize}

\subsection{Forward euler method}
$$
\psi(t + \Delta t) = (1 - I \Delta t h) \psi(t)
$$

In numerical methods two things we should take care of:
\begin{itemize}
\item the local time discrete error
\item the numerical stability, numerical stability means the result are bounded
\end{itemize}
For the Forward euler method, the local time discrete error is $O(\Delta ^2 t)$
$$
\psi(t + \Delta t) = (1 - I \Delta t h)\psi(t) + O(\Delta t^2)
$$
We say this method is order One.

Numerical stability requires
$$
|(1 - I \Delta t h)\psi(t)| < (1 + \epsilon)|\psi(t)|
$$
For every $\psi(t)$. for $\epsilon$ we need
$$
(1 + \epsilon)^{T/\Delta t} - 1 << 1
$$
or
$$
\epsilon \ll \Delta t / T
$$
we have
$$
|(1 - I \Delta t h)\psi(t)| < (1 + \Delta t / T)|\psi(t)|
$$

 For a eigenstate of eigenenergy of $e$, we have
$$
| 1 - I \Delta t e| \le (1 + \Delta t / T)
$$
$$
(\Delta t e)^2 \le 2 \Delta t / T
$$

Because $T >> \Delta t$, so
$$
(\Delta t e)^2 << 1
$$

For a discrete eystem. the max energy is $\hbar^2/2m/(\Delta x)^2$. So we need
$$
\Delta t < (\hbar^2/2m(\Delta x)^2)^{-1}
$$
The requrement for $\Delta t$ is too strict.

\subsection{Backward Euler method}

$$
\psi(t + \Delta t) = (1 + I \Delta t h)^{-1} \psi(t)
$$

This is an order of One method
The stability requres
$$
|(1 + I \Delta t h)^{-1}\psi| < (1 + \Delta t / T) \psi
$$
Because for each eigenstate with eigenernery of $e$
$$
|(1 + I \Delta t e)^{-1}| < 1 < (1 + \Delta t / T)
$$
This methods is unconditional stable

\subsection{Gauss-Legendre Methods}

Gauss-Legendre methods are implicit Runge-Kutta methods.

Let's take the Gauss-legendre method of order two as an exmaple. The Gauss-legendre method of order two is also called implicit midpoint method.
We don't decribe the details of Gauss-Legendre Methods. We only conclude some key points here.
The $\psi(t+\Delta t)$ is given by
$$
\psi(t+\Delta t) = (1+k_1\Delta t) \psi(t)
$$
where $k_1$ should be solved from the following equation
$$
k_1 = - I h(t+1/2\Delta t) (1 + 1/2 k_1\Delta t)
$$
The solution is
$$
k_1 = ((-I h(t+1/2\Delta t))^{-1} - 1/2 \Delta t)^{-1}
$$
finally we have
$$
\psi(t+\Delta t) = \frac{1 - 1/2 I \Delta t h(t+1/2\Delta t)} {1 + 1/2 I \Delta t h(t+1/2\Delta t)} \psi(t)
$$

When $h$ is time independent the evolution is simplified as
$$
\psi(t + \Delta t) = \frac{1 - 1/2 I \Delta t h} {1 + 1/2 I \Delta t h} \psi(t)
$$

With the result we can directly draw some conclusion

\begin{itemize}
  \item The transform of each step is unitary matrix, so that the probability and energy are conserved.
  \item This is an order of two method.
$$
\frac{1 - 1/2 I \Delta t h} {1 + 1/2 I \Delta t h}
= 1- I \Delta t h - 1/2(h\Delta t)^2+ 1/4 I(h\Delta t)^3 + O((h\Delta t)^4)
$$
\end{itemize}

We can repeat the process for Gauss-legendre method of order of four or six. we have the following results
\begin{align*}
\psi(t + \Delta t) &= \frac{1 - 1/2 I \Delta t h - 1/12 (\Delta t h)^2}{ 1 + 1/2 I \Delta t h - 1/12 (\Delta t h)^2} \psi(t)\\
&=(1 - I h \Delta t - 1/2 (h \Delta t)^2 + 1/6 (h \Delta t)^3 + 1/24(h \Delta t)^4 - 1/144(h \Delta t)^5 + O((h \Delta t)^6)\psi(t)
\end{align*}

\begin{align*}
\psi(t + \Delta t) &= \frac{1 - 1/2 I \Delta t h - 1/10 (\Delta t h)^2 + 1/120 I (\Delta t h)^3}{ 1 + 1/2 I \Delta t h - 1/10 (\Delta t h)^2 - 1/120 I (\Delta t h)^3} \psi(t)\\
&=(1 - I h \Delta t - 1/2 (h \Delta t)^2 + 1/6 I(h \Delta t)^3 + 1/24(h \Delta t)^4 - 1/120I(h \Delta t)^5\\
&-1/720(h \Delta t)^6 + 1/4800I(h \Delta t)^7+ O((h \Delta t)^7)\psi(t)
\end{align*}
Note the exactly expansion is
\begin{multline}
\exp^{-Ih\Delta t} = 1 - I h \Delta t - 1/2 (h \Delta t)^2 + 1/6 I(h \Delta t)^3 + 1/24(h \Delta t)^4 - 1/120I(h \Delta t)^5\\
-1/720(h \Delta t)^6 + 1/5040I(h \Delta t)^7+ O((h \Delta t)^7)
\end{multline}
The `GaussLegendreOrderFour.wl` and  `GaussLegendreOrderSix.wl` finish the detailed work.
We may have alternative method, For example
\begin{align*}
&\frac{1 - 1/2 I \Delta t h - 1/12 (\Delta t h)^2}{ 1 + 1/2 I \Delta t h - 1/12 (\Delta t h)^2}\\
&=\frac{(1-1/12(3I-\sqrt{3})h \Delta t)(1-1/12(3I+\sqrt{3})h \Delta t)}{(1-1/12(-3I-\sqrt{3})h \Delta t)(1-1/12(-3I+\sqrt{3})h \Delta t)}\\
&=\frac{(1 - 1/2 I \Delta t h - 1/12 (\Delta t h)^2}{(1-1/12(-3I-\sqrt{3})h \Delta t)(1-1/12(-3I+\sqrt{3})h \Delta t)}
\end{align*}
The advantage is that, the $h^2$ term may introduce some problems, in the method later, we only need to inverse two matrix which are relative simple.

\section{Initial Value Problem}
For the initial value problem, the state at a point is a complex vector of dimension of two
$$
y(x)=\begin{pmatrix} \psi(x) \\ \psi^\prime(x) \end{pmatrix}
$$
The schrodinger equation in matrix form is
$$
\df y = A(x) y
$$
where
$$
A(x)=\df x\begin{pmatrix}  0 & 1 \\ -x\frac{2m}{\hbar^2}(E-V(x)) & 0 \end{pmatrix}
$$
Direct use of range-kutta method will be good choice.
The only different from the time evolution problem is that, the potential will almost depend on the location, however, the hamiltonian will not depend on the time in a wide classes of case.
We define the $T$ matrix as
$$
y(x +\Delta x) = T y(x)
$$

The implementation for explicit methods will be very straight forward. for the implicit it is worth to do some calculation manualy for the speed and numerical accuray.
For the implicit midpoint method
\begin{align*}
K &= A(1 + 1/2K)\\
T &= 1 + K
\end{align*}
where
$$
A = A(x+1/2\Delta x) =\Delta x \begin{pmatrix}  0 & 1 \\ -\frac{2m}{\hbar^2}(E-V(x+1/2\Delta x)) & 0 \end{pmatrix}
$$
Thus
$$
T = 1 + \frac{A}{1 - 1/2A}
$$

For the gauss-legendre of order of four
\begin{align*}
K_1 &= A_1(1 + c_{11} K_1 + c_{12} K_2)\\
K_2 &= A_2(1 + c_{21} K_1 + c_{22} K_2)\\
T &= 1 + 1/2(K_1+K_2)
\end{align*}
where
\begin{align*}
A_1 = A(x + c_1\Delta x)\\
A_2 = A(x + c_2\Delta x)
\end{align*}
The solution is
$$
T = 1 + c_1 (1 - a_{22} A_2 - a_{11} A_1 + (a_{22} a_{11} - a_{12} a_{21})A_2 A_1)^{-1}(A_1+(a_{12}-a_{22})A_2A_1) + (1\leftrightarrow 2)
$$
a delicate program is deveped for the matrix of dimension of two. but the program can never be smart as man. the more work we did, the lesser calculation the computer need to do on run time. please simplify the $A_2A_1$ as a excise. and note $A_2A_1\neq A_1A_2$.
\subsection{Small round error}
Finally we need to calculate
$$
T_1 T_2 T_3 ...
$$
each $T_i$ is very close to unitary. We will lose some precision, because only the difference of $T_i$ from 1 carry the information. The use full information were trucked in the floating architecture.

Let's focus the 
$$
T \leftarrow T T_i
$$
The real change of $T$ can be precisisely calculated from
$$
T(T_i - 1)
$$
However the change we really store in $T$ is
$$
(T T_i) - T
$$
Latter is less precise, because of large round error. We can reduce the effect of round error by recording the difference. 
$$
T(T_i -1) - ((T T_i) - T)
$$
The new method is to use two matrix. one is for the main part, the other one is for the little part.
\begin{align}
\text{BIG} &\leftarrow \text{BIG} T_i\\
\text{LITTLE} &\leftarrow \text{LITTLE} T_i +  \text{BIG}(T_i -1) - ((\text{BIG} T_i) - \text{BIG})
\end{align}
And this is reason that we represent our result in the form $T_i = 1 + ...$ without any implication. We need calculate $T_i-1$ directly.

\section{Perburbation}

The fermi golden rule
$$
\Gamma_{i\rightarrow f} = \frac{2\pi}{\hbar} \rho_E |\bra{\psi_f}V\ket{\psi_i}|^2
$$

\subsection{3D dimential}

In 3D,
$$
\rho_E = \frac{
\frac{V}{(2\pi\hbar)^3}{ p^2 d\Omega dp}
}{
\frac{p dp}{m}
}
=
\frac{m V}{(2\pi\hbar)^3}{ p d\Omega}
$$
Box normalization:
$$
\psi_i(r) = \frac{1}{\sqrt{V}} e^{i k_i r}
$$

$$
\psi_i(r) = \frac{1}{\sqrt{V}} e^{i k_f r}
$$

$$
\bra{\psi_f} V(r) \ket{\psi_i} =  \frac{1}{V} \int dr V(r) e^{i(k_f - k_i) r}
$$

$$
\Gamma_{i\rightarrow f} = \frac{mp d\Omega}{(2\pi \hbar^2)^2 V} |\int dr V(r) e^{i(k_f - k_i) r}|^2
$$

$$
\Gamma_{i\rightarrow f} = \frac{d\sigma}{d\Omega} \frac{p}{m} |\psi_i|^2 d\Omega
$$

$$
\frac{d\sigma}{d\Omega} = \frac{m^2}{(2\pi \hbar^2)^2} |\int dr V(r) e^{i(k_f - k_i) r}|^2
$$

For $V(r) = V_0 e^{-\alpha r^2}$

$$
\int dr V(r) e^{i(k_f - k_i) r} = V_0 (\frac{\pi}{\alpha})^{3/2} e^{-(k_i - k_f)^2/(4\alpha)}
$$


\subsection{2D dimential}

In 2D,
$$
\rho_E = \frac{
\frac{V}{(2\pi\hbar)^2}{ p d\Omega dp}
}{
\frac{p dp}{m}
}
=
\frac{m V}{(2\pi\hbar)^2}{d\Omega}
$$
Box normalization:
$$
\psi_i(r) = \frac{1}{\sqrt{V}} e^{i k_i r}
$$

$$
\psi_i(r) = \frac{1}{\sqrt{V}} e^{i k_f r}
$$

$$
\bra{\psi_f} V(r) \ket{\psi_i} =  \frac{1}{V} \int dr V(r) e^{i(k_f - k_i) r}
$$

$$
\Gamma_{i\rightarrow f} = \frac{m d\Omega}{2\pi \hbar^3 V} |\int dr V(r) e^{i(k_f - k_i) r}|^2
$$

$$
\Gamma_{i\rightarrow f} = \frac{d\sigma}{d\Omega} \frac{p}{m} |\psi_i|^2 d\Omega
$$

$$
\frac{d\sigma}{d\Omega} = \frac{m^2}{2\pi \hbar^3 p} |\int dr V(r) e^{i(k_f - k_i) r}|^2
$$

For $V(r) = V_0 e^{-\alpha r^2}$

$$
\int dr V(r) e^{i(k_f - k_i) r} = \frac{V_0 \pi}{\alpha} e^{-(k_i - k_f)^2/(4\alpha)}
$$

\end{document}
