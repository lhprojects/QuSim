    \documentclass[12pt,twoside]{article}
    \usepackage{amsmath}
    \usepackage{amssymb}
    \usepackage{amsthm}
    \usepackage{amsfonts}
    \usepackage{braket}

\begin{document}

\title{LaTeX Typesetting By Example}
\author{Phil Farrell\\
Stanford University School of Earth Sciences}

\renewcommand{\today}{November 2, 1994}
\maketitle

$$
-\hbar^2/(2m) \nabla^2 \psi = i \hbar \partial_t \psi
$$

$$
\psi_p(x) = e^{ i/\hbar (p x - p^2 /(2m)t)}
$$
$$
\Psi(x) = \int dp \phi_x(p) \psi_p(x)
$$

Note
$$
\int dx \psi_{p^\prime}^*(x) \psi_{p}(x) = e^{ i/\hbar ((p^{\prime2} - p^2) /(2m)t)} 2\pi\delta(1/\hbar(p-p^\prime))
=2\pi\hbar\delta(p-p^\prime)
$$

$$
\phi_x(p) = \frac{1}{2\pi\hbar}\int dx \psi^*_p(x) \Psi(x)
$$
the solution for any time:
$$
\Psi(x, t) = \int dp \frac{1}{2\pi\hbar}\int dy \psi^*_p(y) \Psi(y) \psi_p(x,t)
$$

$$
\Psi(x, t) = \int dy  \Psi(y) \int dp \frac{1}{2\pi\hbar}\psi^*_p(y) \psi_p(x,t)
$$

$$
\Psi(x, t) = \int dy  \Psi(y) \int dp \frac{1}{2\pi\hbar} e^{i(p(x-y)/\hbar - p^2 t/(2m\hbar))}
$$

$$
T=t/(m\hbar)
$$
$$
\Psi(x, t) = \int dy  \Psi(y) \frac{1}{2\pi\hbar} (1-i) e^{i (x-y)^2 / (2 T \hbar^2)}  \sqrt{\pi / T}
$$


For space $a$
$$
-\hbar^2/(2m a^2) (\psi(x+a) + \psi(x-a) - 2\psi(x)) \psi = i \hbar \partial_t \psi
$$

$$
\psi_p(x) = e^{ i/\hbar (p x - \hbar^2(1 - \cos(a p/\hbar))/(m a^2)t)}
$$

$$
\psi_p(x) = e^{ i/\hbar (p x - (p^2/(2m))t)}
$$

$$
0 \le p < 2\pi / a
$$

$$
\sum_{x_i} \psi_{p^\prime}^*(x_i) \psi_{p}(x_i) = e^{ i/\hbar (\hbar^2(1 - \cos(a p^\prime/\hbar) - \hbar^2(1 - \cos(a p/\hbar))/(m a^2)t)}
\sum_{x_i}e^{i(p-p^\prime)x_i/\hbar} = 2\pi\delta(1/\hbar(p - p\prime))
$$
$$
x_i = ai
$$

$$
\phi_x(p) = \frac{1}{2\pi\hbar}\sum_{x_i} \psi^*_p(x_i) \Psi(x_i)
$$

$$
\Psi(x, t) = \sum_{y_i} \Psi(y_i) \int dp \frac{1}{2\pi\hbar} e^{ ip(x-y)/\hbar - i\hbar(1 - \cos(a p/\hbar))/(m a^2)t}
$$



$$
g(k) = \int dx f(x) \exp(ikx)
$$

$$
f(x) = \frac{1}{2\pi}\int dk g(k) \exp(-ikx)
$$

$$
f(x) = \sum_j \delta(x - ja)
$$
$$
g(k) = \int dx \sum_j \delta(x - ja) \exp(ikx) = \sum_j \exp(ijka)
$$

$$
f(x) = \sum_j \delta(x - ja) = \frac{1}{2\pi}\int dk g(k) \exp(-ikx)
$$

$$
g(k) = 1/(2\pi a)\sum_n \delta(k+2\pi n/a)
$$


\section{Infinite Wall}

Eigen state
$$
\psi_k(j) = \sin(jk\pi/N)
$$

$$
j=0..N-1
$$

$$
k=1...N-1
$$

Eigen value
$$
E_k = k\pi/(N a)
$$

$$
\sum_j \psi_j(k)\psi_j(k) = N/2
$$

$$
\Psi(j) = \sum_{k=1}^{N-1} \Phi(k) \psi_k(j)
$$

$$
\Phi(k) = (2/N) \sum_j \Psi(j) \psi_k(j)
$$

$$
\Phi(k) = (2/N) \sum_j \Psi(j) \sin(jk\pi/N)
$$

$$
\Phi(k) = (1/(Ni)) \sum_j \Psi(j) (\exp(ijk\pi/N) - \exp(-ijk\pi/N))
$$

$$
\Phi(k) = (1/(Ni)) \sum_j \Psi(j) (\exp(ijk2\pi/2N) - \exp(i(2N-j)k2\pi/2N))
$$

$$
\Phi(k) = \frac{1}{Ni}\left(\sum_j^{N-1} \Psi(j)\exp(ijk2\pi/2N) - \sum_{N+1}^{2N} \Psi(2N-j)\exp(jk2\pi/2N))\right)
$$

by define
$$
\Phi(N)=0
$$

$$
\Phi(k) = \frac{1}{Ni}\left(\sum_j^{N-1} \Psi(j)\exp(ijk2\pi/2N) - \sum_{j=N}^{2N-1} \Psi(2N-j)\exp(ijk2\pi/2N))\right)
$$

$$
\Phi(k) = \frac{1}{Ni}\left(\sum_{j=0}^{2N-1} \Psi^\prime(j)\exp(ijk2\pi/2N)\right)
$$

$$
\Psi^\prime(j) = \Psi(j) (j \le N-1)
$$
$$
\Psi^\prime(j) = 0 (j = N)
$$
$$
\Psi^\prime(j) = -\Psi(2N-j) (j > N)
$$

$$
\Phi(k) = \frac{1}{Ni} {\rm inv fft}(2N, \Psi^\prime(j), k)
$$


$$
\Psi(j) = \sum_{k=1}^{N-1} \Phi(k) \sin(jk\pi/N)
$$

$$
\Psi(j) = \frac{1}{2i} {\rm inv fft}(2N, \Phi^\prime(k), j)
$$

\section{Split Method}

Second order approximation
$$
e^{t+v} = e^{1/2 v} e^{ t} e^{1/2 v}
$$

Fourth-order approximation
$$
e^{t+v} = e^{c_1 v} e^{d_1 t} e^{c_2 v} e^{d_2 t} e^{ c_2 v} e^{ d_1 t} e^{ c_1 v}
$$

where
$$
c_1 = 1/(2(2-2^{1/3}))
$$
$$
c_2 = (1-2^{1/3})/(2(2-2^{1/3}))
$$
$$
d_1 = 1/(2-2^{1/3})
$$
$$
d_2 = -2^{1/3}/(2-2^{1/3})
$$

\section{Eigen Method}

\section{Gauss-Legendre Method}
The SE, can be written as
$$
\frac{d\psi}{ dt} = - I h \psi
$$
the formal solution is
$$
\psi(t) = e^{ -I h t}\psi(0)
$$

In numerical analysis, many methods are special cases of Runge-Kutta methods.
\begin{itemize}
\item Euler Method
\begin{itemize}
    \item Forward Euler Method
    \item Backward Euler Method
\end{itemize}
\item Gauss-Legendre Method
\item ...
\end{itemize}

\subsection{Forward euler method}
$$
\psi(t + \Delta t) = (1 - I \Delta t h) \psi(t)
$$

In numerical methods two things we should take care of:
\begin{itemize}
\item the local time discrete error
\item the numerical stability, numerical stability means the result are bounded
\end{itemize}
For the Forward euler method, the local time discrete error is $O(\Delta ^2 t)$
$$
\psi(t + \Delta t) = (1 - I \Delta t h)\psi(t) + O(\Delta t^2)
$$
We say this method is order One.

Numerical stability requires
$$
|(1 - I \Delta t h)\psi(t)| < (1 + \epsilon)|\psi(t)|
$$
For every $\psi(t)$. for $\epsilon$ we need
$$
(1 + \epsilon)^{T/\Delta t} - 1 << 1
$$
or
$$
\epsilon \ll \Delta t / T
$$
we have
$$
|(1 - I \Delta t h)\psi(t)| < (1 + \Delta t / T)|\psi(t)|
$$

 For a eigenstate of eigenenergy of $e$, we have
$$
| 1 - I \Delta t e| \le (1 + \Delta t / T)
$$
$$
(\Delta t e)^2 \le 2 \Delta t / T
$$

Because $T >> \Delta t$, so
$$
(\Delta t e)^2 << 1
$$

For a discrete eystem. the max energy is $\hbar^2/2m/(\Delta x)^2$. So we need
$$
\Delta t < (\hbar^2/2m(\Delta x)^2)^{-1}
$$
The requrement for $\Delta t$ is too strict.

\subsection{Backward Euler method}

$$
\psi(t + \Delta t) = (1 + I \Delta t h)^{-1} \psi(t)
$$

This is an order of One method
The stability requres
$$
|(1 + I \Delta t h)^{-1}\psi| < (1 + \Delta t / T) \psi
$$
Because for each eigenstate with eigenernery of $e$
$$
|(1 + I \Delta t e)^{-1}| < 1 < (1 + \Delta t / T)
$$
This methods is unconditional stable

\subsection{Gauss-Legendre Methods}

Gauss-Legendre methods are implicit Runge-Kutta methods.

Let's take the Gauss-legendre method of order two as an exmaple. The Gauss-legendre method of order two is also called implicit midpoint method.
We don't decribe the details of Gauss-Legendre Methods. We only conclude some key points here.
The $\psi(t+\Delta t)$ is given by
$$
\psi(t+\Delta t) = (1+k_1\Delta t) \psi(t)
$$
where $k_1$ should be solved from the following equation
$$
k_1 = - I h(t+1/2\Delta t) (1 + 1/2 k_1\Delta t)
$$
The solution is
$$
k_1 = ((-I h(t+1/2\Delta t))^{-1} - 1/2 \Delta t)^{-1}
$$
finally we have
$$
\psi(t+\Delta t) = \frac{1 - 1/2 I \Delta t h(t+1/2\Delta t)} {1 + 1/2 I \Delta t h(t+1/2\Delta t)} \psi(t)
$$

When $h$ is time independent the evolution is simplified as
$$
\psi(t + \Delta t) = \frac{1 - 1/2 I \Delta t h} {1 + 1/2 I \Delta t h} \psi(t)
$$

With the result we can directly draw some conclusion

\begin{itemize}
  \item The transform of each step is unitary matrix, so that the probability and energy are conserved.
  \item This is an order of two method.
$$
\frac{1 - 1/2 I \Delta t h} {1 + 1/2 I \Delta t h}
= 1- I \Delta t h - 1/2(h\Delta t)^2+ 1/4 I(h\Delta t)^3 + O((h\Delta t)^4)
$$
\end{itemize}

We can repeat the process for Gauss-legendre method of order of four or six. we have the following results
\begin{align*}
\psi(t + \Delta t) &= \frac{1 - 1/2 I \Delta t h - 1/12 (\Delta t h)^2}{ 1 + 1/2 I \Delta t h - 1/12 (\Delta t h)^2} \psi(t)\\
&=(1 - I h \Delta t - 1/2 (h \Delta t)^2 + 1/6 (h \Delta t)^3 + 1/24(h \Delta t)^4 - 1/144(h \Delta t)^5 + O((h \Delta t)^6)\psi(t)
\end{align*}

\begin{align*}
\psi(t + \Delta t) &= \frac{1 - 1/2 I \Delta t h - 1/10 (\Delta t h)^2 + 1/120 I (\Delta t h)^3}{ 1 + 1/2 I \Delta t h - 1/10 (\Delta t h)^2 - 1/120 I (\Delta t h)^3} \psi(t)\\
&=(1 - I h \Delta t - 1/2 (h \Delta t)^2 + 1/6 I(h \Delta t)^3 + 1/24(h \Delta t)^4 - 1/120I(h \Delta t)^5\\
&-1/720(h \Delta t)^6 + 1/4800I(h \Delta t)^7+ O((h \Delta t)^7)\psi(t)
\end{align*}
Note the exactly expansion is
\begin{multline}
\exp^{-Ih\Delta t} = 1 - I h \Delta t - 1/2 (h \Delta t)^2 + 1/6 I(h \Delta t)^3 + 1/24(h \Delta t)^4 - 1/120I(h \Delta t)^5\\
-1/720(h \Delta t)^6 + 1/5040I(h \Delta t)^7+ O((h \Delta t)^7)
\end{multline}
The `GaussLegendreOrderFour.wl` and  `GaussLegendreOrderSix.wl` finish the detailed work.
We may have alternative method, For example
\begin{align*}
&\frac{1 - 1/2 I \Delta t h - 1/12 (\Delta t h)^2}{ 1 + 1/2 I \Delta t h - 1/12 (\Delta t h)^2}\\
&=\frac{(1-1/12(3I-\sqrt{3})h \Delta t)(1-1/12(3I+\sqrt{3})h \Delta t)}{(1-1/12(-3I-\sqrt{3})h \Delta t)(1-1/12(-3I+\sqrt{3})h \Delta t)}\\
&=\frac{(1 - 1/2 I \Delta t h - 1/12 (\Delta t h)^2}{(1-1/12(-3I-\sqrt{3})h \Delta t)(1-1/12(-3I+\sqrt{3})h \Delta t)}
\end{align*}
The advantage is that, the $h^2$ term may introduce some problems, in the method later, we only need to inverse two matrix which are relative simple.

\section{Perburbation}

The fermi golden rule
$$
\Gamma_{i\rightarrow f} = \frac{2\pi}{\hbar} \rho_E |\bra{\psi_f}V\ket{\psi_i}|^2
$$

\subsection{3D dimential}

In 3D,
$$
\rho_E = \frac{
\frac{V}{(2\pi\hbar)^3}{ p^2 d\Omega dp}
}{
\frac{p dp}{m}
}
=
\frac{m V}{(2\pi\hbar)^3}{ p d\Omega}
$$
Box normalization:
$$
\psi_i(r) = \frac{1}{\sqrt{V}} e^{i k_i r}
$$

$$
\psi_i(r) = \frac{1}{\sqrt{V}} e^{i k_f r}
$$

$$
\bra{\psi_f} V(r) \ket{\psi_i} =  \frac{1}{V} \int dr V(r) e^{i(k_f - k_i) r}
$$

$$
\Gamma_{i\rightarrow f} = \frac{mp d\Omega}{(2\pi \hbar^2)^2 V} |\int dr V(r) e^{i(k_f - k_i) r}|^2
$$

$$
\Gamma_{i\rightarrow f} = \frac{d\sigma}{d\Omega} \frac{p}{m} |\psi_i|^2 d\Omega
$$

$$
\frac{d\sigma}{d\Omega} = \frac{m^2}{(2\pi \hbar^2)^2} |\int dr V(r) e^{i(k_f - k_i) r}|^2
$$

For $V(r) = V_0 e^{-\alpha r^2}$

$$
\int dr V(r) e^{i(k_f - k_i) r} = V_0 (\frac{\pi}{\alpha})^{3/2} e^{-(k_i - k_f)^2/(4\alpha)}
$$


\subsection{2D dimential}

In 2D,
$$
\rho_E = \frac{
\frac{V}{(2\pi\hbar)^2}{ p d\Omega dp}
}{
\frac{p dp}{m}
}
=
\frac{m V}{(2\pi\hbar)^2}{d\Omega}
$$
Box normalization:
$$
\psi_i(r) = \frac{1}{\sqrt{V}} e^{i k_i r}
$$

$$
\psi_i(r) = \frac{1}{\sqrt{V}} e^{i k_f r}
$$

$$
\bra{\psi_f} V(r) \ket{\psi_i} =  \frac{1}{V} \int dr V(r) e^{i(k_f - k_i) r}
$$

$$
\Gamma_{i\rightarrow f} = \frac{m d\Omega}{2\pi \hbar^3 V} |\int dr V(r) e^{i(k_f - k_i) r}|^2
$$

$$
\Gamma_{i\rightarrow f} = \frac{d\sigma}{d\Omega} \frac{p}{m} |\psi_i|^2 d\Omega
$$

$$
\frac{d\sigma}{d\Omega} = \frac{m^2}{2\pi \hbar^3 p} |\int dr V(r) e^{i(k_f - k_i) r}|^2
$$

For $V(r) = V_0 e^{-\alpha r^2}$

$$
\int dr V(r) e^{i(k_f - k_i) r} = \frac{V_0 \pi}{\alpha} e^{-(k_i - k_f)^2/(4\alpha)}
$$

\end{document}
