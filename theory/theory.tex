\documentstyle[11pt]{article}

\begin{document}

\title{LaTeX Typesetting By Example}
\author{Phil Farrell\\
Stanford University School of Earth Sciences}

\renewcommand{\today}{November 2, 1994}
\maketitle

$$
-\hbar^2/(2m) \nabla^2 \psi = i \hbar \partial_t \psi
$$

$$
\psi_p(x) = e^{ i/\hbar (p x - p^2 /(2m)t)}
$$
$$
\Psi(x) = \int dp \phi_x(p) \psi_p(x)
$$

Note
$$
\int dx \psi_{p^\prime}^*(x) \psi_{p}(x) = e^{ i/\hbar ((p^{\prime2} - p^2) /(2m)t)} 2\pi\delta(1/\hbar(p-p^\prime))
=2\pi\hbar\delta(p-p^\prime)
$$

$$
\phi_x(p) = \frac{1}{2\pi\hbar}\int dx \psi^*_p(x) \Psi(x)
$$
the solution for any time:
$$
\Psi(x, t) = \int dp \frac{1}{2\pi\hbar}\int dy \psi^*_p(y) \Psi(y) \psi_p(x,t)
$$

$$
\Psi(x, t) = \int dy  \Psi(y) \int dp \frac{1}{2\pi\hbar}\psi^*_p(y) \psi_p(x,t)
$$

$$
\Psi(x, t) = \int dy  \Psi(y) \int dp \frac{1}{2\pi\hbar} e^{i(p(x-y)/\hbar - p^2 t/(2m\hbar))}
$$

$$
T=t/(m\hbar)
$$
$$
\Psi(x, t) = \int dy  \Psi(y) \frac{1}{2\pi\hbar} (1-i) e^{i (x-y)^2 / (2 T \hbar^2)}  \sqrt{\pi / T}
$$


For space $a$
$$
-\hbar^2/(2m a^2) (\psi(x+a) + \psi(x-a) - 2\psi(x)) \psi = i \hbar \partial_t \psi
$$

$$
\psi_p(x) = e^{ i/\hbar (p x - \hbar^2(1 - \cos(a p/\hbar))/(m a^2)t)}
$$

$$
\psi_p(x) = e^{ i/\hbar (p x - (p^2/(2m))t)}
$$

$$
0 \le p < 2\pi / a
$$

$$
\sum_{x_i} \psi_{p^\prime}^*(x_i) \psi_{p}(x_i) = e^{ i/\hbar (\hbar^2(1 - \cos(a p^\prime/\hbar) - \hbar^2(1 - \cos(a p/\hbar))/(m a^2)t)}
\sum_{x_i}e^{i(p-p^\prime)x_i/\hbar} = 2\pi\delta(1/\hbar(p - p\prime))
$$
$$
x_i = ai
$$

$$
\phi_x(p) = \frac{1}{2\pi\hbar}\sum_{x_i} \psi^*_p(x_i) \Psi(x_i)
$$

$$
\Psi(x, t) = \sum_{y_i} \Psi(y_i) \int dp \frac{1}{2\pi\hbar} e^{ ip(x-y)/\hbar - i\hbar(1 - \cos(a p/\hbar))/(m a^2)t}
$$



$$
g(k) = \int dx f(x) \exp(ikx)
$$

$$
f(x) = \frac{1}{2\pi}\int dk g(k) \exp(-ikx)
$$

$$
f(x) = \sum_j \delta(x - ja)
$$
$$
g(k) = \int dx \sum_j \delta(x - ja) \exp(ikx) = \sum_j \exp(ijka)
$$

$$
f(x) = \sum_j \delta(x - ja) = \frac{1}{2\pi}\int dk g(k) \exp(-ikx)
$$

$$
g(k) = 1/(2\pi a)\sum_n \delta(k+2\pi n/a)
$$


\section{Infinite Wall}

Eigen state
$$
\psi_k(j) = \sin(jk\pi/N)
$$

$$
j=0..N-1
$$

$$
k=1...N-1
$$

Eigen value
$$
E_k = k\pi/(N a)
$$

$$
\sum_j \psi_j(k)\psi_j(k) = N/2
$$

$$
\Psi(j) = \sum_{k=1}^{N-1} \Phi(k) \psi_k(j)
$$

$$
\Phi(k) = (2/N) \sum_j \Psi(j) \psi_k(j)
$$

$$
\Phi(k) = (2/N) \sum_j \Psi(j) \sin(jk\pi/N)
$$

$$
\Phi(k) = (1/(Ni)) \sum_j \Psi(j) (\exp(ijk\pi/N) - \exp(-ijk\pi/N))
$$

$$
\Phi(k) = (1/(Ni)) \sum_j \Psi(j) (\exp(ijk2\pi/2N) - \exp(i(2N-j)k2\pi/2N))
$$

$$
\Phi(k) = \frac{1}{Ni}\left(\sum_j^{N-1} \Psi(j)\exp(ijk2\pi/2N) - \sum_{N+1}^{2N} \Psi(2N-j)\exp(jk2\pi/2N))\right)
$$

by define
$$
\Phi(N)=0
$$

$$
\Phi(k) = \frac{1}{Ni}\left(\sum_j^{N-1} \Psi(j)\exp(ijk2\pi/2N) - \sum_{j=N}^{2N-1} \Psi(2N-j)\exp(ijk2\pi/2N))\right)
$$

$$
\Phi(k) = \frac{1}{Ni}\left(\sum_{j=0}^{2N-1} \Psi^\prime(j)\exp(ijk2\pi/2N)\right)
$$

$$
\Psi^\prime(j) = \Psi(j) (j \le N-1)
$$
$$
\Psi^\prime(j) = 0 (j = N)
$$
$$
\Psi^\prime(j) = -\Psi(2N-j) (j > N)
$$

$$
\Phi(k) = \frac{1}{Ni} {\rm inv fft}(2N, \Psi^\prime(j), k)
$$


$$
\Psi(j) = \sum_{k=1}^{N-1} \Phi(k) \sin(jk\pi/N)
$$

$$
\Psi(j) = \frac{1}{2i} {\rm inv fft}(2N, \Phi^\prime(k), j)
$$

\section{Split Method}

Second order approximation
$$
e^{t+v} = e^{1/2 v} e^{ t} e^{1/2 v}
$$

Fourth-order approximation
$$
e^{t+v} = e^{c_1 v} e^{d_1 t} e^{c_2 v} e^{d_2 t} e^{ c_2 v} e^{ d_1 t} e^{ c_1 v}
$$

where
$$
c_1 = 1/(2(2-2^{1/3}))
$$
$$
c_2 = (1-2^{1/3})/(2(2-2^{1/3}))
$$
$$
d_1 = 1/(2-2^{1/3})
$$
$$
d_2 = -2^{1/3}/(2-2^{1/3})
$$

\section{Eigen Method}

\section{Gauss-Legendre Method}
The SE, can be written as
$$
\frac{d\psi}{ dt} = - I h \psi
$$
the formal solution is
$$
\psi(t) = e^{ -I h t}\psi(0)
$$

In numerical analysis, many methods are special cases of Runge-Kutta methods.
\begin{itemize}
\item Euler Method
\begin{itemize}
    \item Forward Euler Method
    \item Backward Euler Method
\end{itemize}
\item Gauss-Legendre Method
\item ...
\end{itemize}

\subsection{Forward euler method}
$$
\psi(t + \Delta t) = (1 - I \Delta t h) \psi(t)
$$

In numerical methods two things we should take care of:
\begin{itemize}
\item the local time discrete error
\item the numerical stability, numerical stability means the result are bounded
\end{itemize}
For the Forward euler method, the local time discrete error is $O(\Delta ^2 t)$
$$
\psi(t + \Delta t) = (1 - I \Delta t h)\psi(t) + O(\Delta t^2)
$$
We say this method is order One.

Numerical stability requires
$$
|(1 - I \Delta t h)\psi(t)| < (1 + \epsilon)|\psi(t)|
$$
For every $\psi(t)$. for $\epsilon$ we need
$$
(1 + \epsilon)^{T/\Delta t} - 1 << 1
$$
or
$$
\epsilon \ll \Delta t / T
$$
we have
$$
|(1 - I \Delta t h)\psi(t)| < (1 + \Delta t / T)|\psi(t)|
$$

 For a eigenstate of eigenenergy of $e$, we have
$$
| 1 - I \Delta t e| \le (1 + \Delta t / T)
$$
$$
(\Delta t e)^2 \le 2 \Delta t / T
$$

Because $T >> \Delta t$, so
$$
(\Delta t e)^2 << 1
$$

For a discrete eystem. the max energy is $\hbar^2/2m/(\Delta x)^2$. So we need
$$
\Delta t < (\hbar^2/2m(\Delta x)^2)^{-1}
$$
The requrement for $\Delta t$ is too strict.

\subsection{Backward Euler method}

$$
\psi(t + \Delta t) = (1 + I \Delta t h)^{-1} \psi(t)
$$

This is an order of One method
The stability requres
$$
|(1 + I \Delta t h)^{-1}\psi| < (1 + \Delta t / T) \psi
$$
Because for each eigenstate with eigenernery of $e$
$$
|(1 + I \Delta t e)^{-1}| < 1 < (1 + \Delta t / T)
$$
This methods is unconditional stable

\subsection{Gauss-Legendre methods}

Gauss-Legendre methods are implicit Runge-Kutta methods.

 The Gauss-legendre method of order two is also called implicit midpoint method.
We don't decribe the details of Gauss-Legendre Methods. We only conclude some key points here.
$$
k_1 = - I h(t_1) (1 + c_{11} k_1\Delta t + c_{12} k_2\Delta t)
$$
$$
k_2 = - I h(t_2) (1 + c_{21} k_1\Delta t + c_{22} k_2\Delta t)
$$
$$
\psi(t+\Delta t) = (1+k_1\Delta t + k_2\Delta t) \psi(t)
$$
$k1$ is sovled as
$$
k_1 = A_1(1-c_{22} A_2\Delta t-c_{11}A_1\Delta t+(c_{22}c_{11}-c_{12}c_{21}) A_1A_2(\Delta t)^2)^{-1}(1+(-c_{22}+c_{12})A_2\Delta t)
$$
$$
k_2 = A_2(1-c_{22}A_2\Delta t -c_{11}A_1\Delta t+(c_{22}c_{11}-c_{12}c_{21}) A_2A_1(\Delta t)^2)^{-1}(1+(-c_{11}+c_{21})A_1\Delta t)
$$
where $A_i=-Ih(t_i)$.


When $h$ is time independent the evolution is simplified as:
$$
\psi(t + \Delta t) = (1 + 1/2 I \Delta t h)^{-1} (1 - 1/2 I \Delta t h) \psi(t)
$$
$$
\psi(t + \Delta t) = \frac{1 - 1/2 I \Delta t h - 1/12 (\Delta t h)^2}{ 1 + 1/2 I \Delta t h - 1/12 (\Delta t h)^2} \psi(t)
$$
$$
\psi(t + \Delta t) = \frac{1 - 1/2 I \Delta t h - 1/10 (\Delta t h)^2 + 1/120 I (\Delta t h)^3}{ 1 + 1/2 I \Delta t h - 1/10 (\Delta t h)^2 - 1/120 I (\Delta t h)^3} \psi(t)
$$

Order of two method.
Stable unconditionally.
The transform each step is unitary matrix, so that the probability and energy are conserved.



\end{document}
