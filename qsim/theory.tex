\documentstyle[11pt]{article}

\begin{document}

\title{LaTeX Typesetting By Example}
\author{Phil Farrell\\
Stanford University School of Earth Sciences}

\renewcommand{\today}{November 2, 1994}
\maketitle

$$
-\hbar^2/(2m) \nabla^2 \psi = i \hbar \partial_t \psi
$$

$$
\psi_p(x) = e^{ i/\hbar (p x - p^2 /(2m)t)}
$$
$$
\Psi(x) = \int dp \phi_x(p) \psi_p(x)
$$

Note
$$
\int dx \psi_{p^\prime}^*(x) \psi_{p}(x) = e^{ i/\hbar ((p^{\prime2} - p^2) /(2m)t)} 2\pi\delta(1/\hbar(p-p^\prime))
=2\pi\hbar\delta(p-p^\prime)
$$

$$
\phi_x(p) = \frac{1}{2\pi\hbar}\int dx \psi^*_p(x) \Psi(x)
$$
the solution for any time:
$$
\Psi(x, t) = \int dp \frac{1}{2\pi\hbar}\int dy \psi^*_p(y) \Psi(y) \psi_p(x,t)
$$

$$
\Psi(x, t) = \int dy  \Psi(y) \int dp \frac{1}{2\pi\hbar}\psi^*_p(y) \psi_p(x,t)
$$

$$
\Psi(x, t) = \int dy  \Psi(y) \int dp \frac{1}{2\pi\hbar} e^{i(p(x-y)/\hbar - p^2 t/(2m\hbar))}
$$

$$
T=t/(m\hbar)
$$
$$
\Psi(x, t) = \int dy  \Psi(y) \frac{1}{2\pi\hbar} (1-i) e^{i (x-y)^2 / (2 T \hbar^2)}  \sqrt{\pi / T}
$$


For space $a$
$$
-\hbar^2/(2m a^2) (\psi(x+a) + \psi(x-a) - 2\psi(x)) \psi = i \hbar \partial_t \psi
$$

$$
\psi_p(x) = e^{ i/\hbar (p x - \hbar^2(1 - \cos(a p/\hbar))/(m a^2)t)}
$$

$$
\psi_p(x) = e^{ i/\hbar (p x - (p^2/(2m))t)}
$$

$$
0 \le p < 2\pi / a
$$

$$
\sum_{x_i} \psi_{p^\prime}^*(x_i) \psi_{p}(x_i) = e^{ i/\hbar (\hbar^2(1 - \cos(a p^\prime/\hbar) - \hbar^2(1 - \cos(a p/\hbar))/(m a^2)t)}
\sum_{x_i}e^{i(p-p^\prime)x_i/\hbar} = 2\pi\delta(1/\hbar(p - p\prime))
$$
$$
x_i = ai
$$

$$
\phi_x(p) = \frac{1}{2\pi\hbar}\sum_{x_i} \psi^*_p(x_i) \Psi(x_i)
$$

$$
\Psi(x, t) = \sum_{y_i} \Psi(y_i) \int dp \frac{1}{2\pi\hbar} e^{ ip(x-y)/\hbar - i\hbar(1 - \cos(a p/\hbar))/(m a^2)t}
$$



$$
g(k) = \int dx f(x) \exp(ikx)
$$

$$
f(x) = \frac{1}{2\pi}\int dk g(k) \exp(-ikx)
$$

$$
f(x) = \sum_j \delta(x - ja)
$$
$$
g(k) = \int dx \sum_j \delta(x - ja) \exp(ikx) = \sum_j \exp(ijka)
$$

$$
f(x) = \sum_j \delta(x - ja) = \frac{1}{2\pi}\int dk g(k) \exp(-ikx)
$$

$$
g(k) = 1/(2\pi a)\sum_n \delta(k+2\pi n/a)
$$


\section{Infinite Wall}

Eigen state
$$
\psi_k(j) = \sin(jk\pi/N)
$$

$$
j=0..N-1
$$

$$
k=1...N-1
$$

Eigen value
$$
E_k = k\pi/(N a)
$$

$$
\sum_j \psi_j(k)\psi_j(k) = N/2
$$

$$
\Psi(j) = \sum_{k=1}^{N-1} \Phi(k) \psi_k(j)
$$

$$
\Phi(k) = (2/N) \sum_j \Psi(j) \psi_k(j)
$$

$$
\Phi(k) = (2/N) \sum_j \Psi(j) \sin(jk\pi/N)
$$

$$
\Phi(k) = (1/(Ni)) \sum_j \Psi(j) (\exp(ijk\pi/N) - \exp(-ijk\pi/N))
$$

$$
\Phi(k) = (1/(Ni)) \sum_j \Psi(j) (\exp(ijk2\pi/2N) - \exp(i(2N-j)k2\pi/2N))
$$

$$
\Phi(k) = \frac{1}{Ni}\left(\sum_j^{N-1} \Psi(j)\exp(ijk2\pi/2N) - \sum_{N+1}^{2N} \Psi(2N-j)\exp(jk2\pi/2N))\right)
$$

by define
$$
\Phi(N)=0
$$

$$
\Phi(k) = \frac{1}{Ni}\left(\sum_j^{N-1} \Psi(j)\exp(ijk2\pi/2N) - \sum_{j=N}^{2N-1} \Psi(2N-j)\exp(ijk2\pi/2N))\right)
$$

$$
\Phi(k) = \frac{1}{Ni}\left(\sum_{j=0}^{2N-1} \Psi^\prime(j)\exp(ijk2\pi/2N)\right)
$$

$$
\Psi^\prime(j) = \Psi(j) (j \le N-1)
$$
$$
\Psi^\prime(j) = 0 (j = N)
$$
$$
\Psi^\prime(j) = -\Psi(2N-j) (j > N)
$$

$$
\Phi(k) = \frac{1}{Ni} {\rm inv fft}(2N, \Psi^\prime(j), k)
$$


$$
\Psi(j) = \sum_{k=1}^{N-1} \Phi(k) \sin(jk\pi/N)
$$

$$
\Psi(j) = \frac{1}{2i} {\rm inv fft}(2N, \Phi^\prime(k), j)
$$


\end{document}
